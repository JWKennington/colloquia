%%%%%%%%%%%%%%%%%%%%%%%%%%%%%%%%%%%%%%%%%%%%%%%%%%%%%
%                                                   %
%     Penn State Colloquium Poster Template         %
%                                                   %
% Uses Penn State Colloquium class, with options:   %
%                                                   %
% Orientation:                                      %
%     portrait (default), landscape                 %
%                                                   %
% Paper size:                                       %
%     a4paper (default), a0paper, a1paper, a2paper, %
%     a3paper, a5paper, a6paper                     %
%%%%%%%%%%%%%%%%%%%%%%%%%%%%%%%%%%%%%%%%%%%%%%%%%%%%%
\documentclass{../psuposter}
\renewcommand{\templateimagepath}{../} 


%%%%%%%%%%%%%%%%%%%%%%%%%%%%%%%%%%%%%%%%%%%%%%%%%%%%%
%               Package Dependencies                %
%%%%%%%%%%%%%%%%%%%%%%%%%%%%%%%%%%%%%%%%%%%%%%%%%%%%%
\usepackage{natbib}
\usepackage{lipsum}                                % Dummy text
\usepackage[figwidth = 0.98\linewidth]{todonotes}  % Dummy image (and more!)
\usepackage[absolute, overlay]{textpos}            % Figure placement
\usepackage{braket}
\setlength{\TPHorizModule}{\paperwidth}
\setlength{\TPVertModule}{\paperheight}
\setcitestyle{numbers,square}


%%%%%%%%%%%%%%%%%%%%%%%%%%%%%%%%%%%%%%%%%%%%%%%%%%%%%
%                 AUTHOR AND TITLE                  %
%%%%%%%%%%%%%%%%%%%%%%%%%%%%%%%%%%%%%%%%%%%%%%%%%%%%%
\title{Moiré Samples: The twisted bilayer graphene scenario}
\author{Andrei Bernevig}
\institute{Princeton University}


%%%%%%%%%%%%%%%%%%%%%%%%%%%%%%%%%%%%%%%%%%%%%%%%%%%%%
%                  BEGIN DOCUMENT                   %
%%%%%%%%%%%%%%%%%%%%%%%%%%%%%%%%%%%%%%%%%%%%%%%%%%%%%
\begin{document}
\begin{frame}
\begin{columns}[t, totalwidth=\textwidth]
\begin{column}{0.45\textwidth - 1cm}


%%%%%%%%%%%%%%%%%%%%%%%%%%%%%%%%%%%%%%%%%%%%%%%%%%%%%
%                 BLOCK: BIOGRAPHY                  %
%%%%%%%%%%%%%%%%%%%%%%%%%%%%%%%%%%%%%%%%%%%%%%%%%%%%%
    \begin{block}{Speaker Biographic Summary}
    	\begin{center}
    		\includegraphics[width=0.68\textwidth]{images/portrait}
    	\end{center}
    	\href{https://phy.princeton.edu/people/bogdan-bernevig}{Dr. Andrei Bernevig} is a Professor of Physics at Princeton University. He earned his doctorate in 2006 at Stanford University and subsequently became a postdoc at the Princeton Center for Theoretical Physics, where he has since held several professorships. Bernevig has received numerous prizes for young researchers, including the 2012 Blavatnik awards for young scientists, the 2014 Sackler Prize, the Packard Fellowship, the 2016 New Horizons Prize, a 2017 Guggenheim Fellowship and most recently the2019 James C. McGroddy Prize for New Materials.

    \end{block}


%%%%%%%%%%%%%%%%%%%%%%%%%%%%%%%%%%%%%%%%%%%%%%%%%%%%%
%            BLOCK: RESEARCH INTERESTS              %
%%%%%%%%%%%%%%%%%%%%%%%%%%%%%%%%%%%%%%%%%%%%%%%%%%%%%
    \begin{block}{Research Interests}
        Dr. Bernevig's research focuses on high-temperature superconductivity in iron-based superconductors. His group is also interested in the field of topological phases and Fractional Quantum Hall effect. 
        \begin{center}
	    	\includegraphics[width=0.65\textwidth]{images/research}    		
	    	
    	\textit{Depiction of the Quantum Hall Effect.} 
    	\end{center}

    	%\cite{longResearchLongLab}

    \end{block}
\end{column}
\begin{column}{0.55\textwidth - 1cm}


%%%%%%%%%%%%%%%%%%%%%%%%%%%%%%%%%%%%%%%%%%%%%%%%%%%%%
%                 BLOCK: ABSTRACT                   %
%%%%%%%%%%%%%%%%%%%%%%%%%%%%%%%%%%%%%%%%%%%%%%%%%%%%%
    \begin{block}{Talk Abstract}
    	Recently, Moire engineered samples have provided a rich and controllable playground of interactions, topology, and superconductivity. In there remarkable systems, superconductivy appears at the lowest ever found electron densities. We present a full theory of the interacting insulating phases of twisted bilayer graphene around the first magic angle where the bandwidth of the “active” bands becomes very small. We show that the single particle Hamiltonian is fully anomalous: it contains stable topology for every set of bands. Furthermore, we analyze the Coulomb interaction and obtain exact insulating ground states as well as the full excitation spectrum in certain limits.
    \end{block}


%%%%%%%%%%%%%%%%%%%%%%%%%%%%%%%%%%%%%%%%%%%%%%%%%%%%%
%                BLOCK: BACKGROUND                  %
%%%%%%%%%%%%%%%%%%%%%%%%%%%%%%%%%%%%%%%%%%%%%%%%%%%%%
    \begin{block}{Brief Background}
    	Prof. Bernevig's work illuminates the relationship of how electrons organize themselves in the matter and the bulk properties of topological materials. Principal calculations involve investigating how materials behave in classic states of matter but also how they transition to superconducting phases. Prof. Bernenvig is particularly focused on topological insulators which are materials that only insulate in their interior whilst conducting electricity on their surface. \cite{wangTopologicalEdgeTransport2021}
    	%\cite{longLocalAxonalConduction2020} 
        \begin{center}
		   	\includegraphics[width=0.85\textwidth]{images/background}
		   	
		   	\textit{Sample layout, electrical transport, and band structure.}\cite{wangTopologicalEdgeTransport2021}    		
    	\end{center}
%		Second Paragraph 
		%\cite{longMorphologicalCharacterizationHVC2018} 
    \end{block}


%%%%%%%%%%%%%%%%%%%%%%%%%%%%%%%%%%%%%%%%%%%%%%%%%%%%%
%                 BLOCK: REFERENCES                 %
%%%%%%%%%%%%%%%%%%%%%%%%%%%%%%%%%%%%%%%%%%%%%%%%%%%%%
    \begin{block}{References}
    \nocite{*}
        \bibliographystyle{aipnum4-1}
%        \bibliographystyle{iopart-num}
		\bibliography{references}
    \end{block}

\end{column}
\end{columns}


%%%%%%%%%%%%%%%%%%%%%%%%%%%%%%%%%%%%%%%%%%%%%%%%%%%%%
%                    FOOTER TEXT                    %
%%%%%%%%%%%%%%%%%%%%%%%%%%%%%%%%%%%%%%%%%%%%%%%%%%%%%
\begin{textblock}{0.5}(0.18, 0.94)
    \color{white}
    \sffamily
    \textbf{Eberly College of Science}
    \\
    Department of Physics
\end{textblock}


%%%%%%%%%%%%%%%%%%%%%%%%%%%%%%%%%%%%%%%%%%%%%%%%%%%%%
%                   END TEMPLATE                    %
%%%%%%%%%%%%%%%%%%%%%%%%%%%%%%%%%%%%%%%%%%%%%%%%%%%%%
\end{frame}
\end{document}
