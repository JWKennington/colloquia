%%%%%%%%%%%%%%%%%%%%%%%%%%%%%%%%%%%%%%%%%%%%%%%%%%%%%
%                                                   %
%     Penn State Colloquium Poster Template         %
%                                                   %
% Uses Penn State Colloquium class, with options:   %
%                                                   %
% Orientation:                                      %
%     portrait (default), landscape                 %
%                                                   %
% Paper size:                                       %
%     a4paper (default), a0paper, a1paper, a2paper, %
%     a3paper, a5paper, a6paper                     %
%%%%%%%%%%%%%%%%%%%%%%%%%%%%%%%%%%%%%%%%%%%%%%%%%%%%%
\documentclass{../psuposter}
\renewcommand{\templateimagepath}{../} 


%%%%%%%%%%%%%%%%%%%%%%%%%%%%%%%%%%%%%%%%%%%%%%%%%%%%%
%               Package Dependencies                %
%%%%%%%%%%%%%%%%%%%%%%%%%%%%%%%%%%%%%%%%%%%%%%%%%%%%%
\usepackage{natbib}
\usepackage{lipsum}                                % Dummy text
\usepackage[figwidth = 0.98\linewidth]{todonotes}  % Dummy image (and more!)
\usepackage[absolute, overlay]{textpos}            % Figure placement
\usepackage{braket}
\setlength{\TPHorizModule}{\paperwidth}
\setlength{\TPVertModule}{\paperheight}
\setcitestyle{numbers,square}


%%%%%%%%%%%%%%%%%%%%%%%%%%%%%%%%%%%%%%%%%%%%%%%%%%%%%
%                 AUTHOR AND TITLE                  %
%%%%%%%%%%%%%%%%%%%%%%%%%%%%%%%%%%%%%%%%%%%%%%%%%%%%%
\title{Quantum Droplets and Supersolidity in a Dipolar Quantum Gas}
\author{Tilman Pfau}
\institute{University of Stuttgart}


%%%%%%%%%%%%%%%%%%%%%%%%%%%%%%%%%%%%%%%%%%%%%%%%%%%%%
%                  BEGIN DOCUMENT                   %
%%%%%%%%%%%%%%%%%%%%%%%%%%%%%%%%%%%%%%%%%%%%%%%%%%%%%
\begin{document}
\begin{frame}
\begin{columns}[t, totalwidth=\textwidth]
\begin{column}{0.45\textwidth - 1cm}


%%%%%%%%%%%%%%%%%%%%%%%%%%%%%%%%%%%%%%%%%%%%%%%%%%%%%
%                 BLOCK: BIOGRAPHY                  %
%%%%%%%%%%%%%%%%%%%%%%%%%%%%%%%%%%%%%%%%%%%%%%%%%%%%%
    \begin{block}{Speaker Biographic Summary}
    	\begin{center}
    		\includegraphics[width=0.6\textwidth]{images/pfau}
    	\end{center}
    	\href{https://www.uni-stuttgart.de/en/press/experts/Prof.-Tilman-Pfau/}{Dr. Tilman Pfau} is a Professor and Chair of Phototonics at the University of Stuttgart. 
    	He completed his Ph.D. in the field of atom optics at the University of Konstanz under the supervision of Prof. Jürgen Mlynek in 1994. 
    	In 2000, he founded a new institute at the University of Stuttgart for the experimental study of interacting many body systems, and he is also co-director of the Center for Integrated Quantum Science and Technology (IQST) in Stuttgart and Ulm. 
%    	He has organized various large-scale outreach activities, and serves as a spokesperson for various research networks. 
    	Prof. Pfau is an elected fellow of the Optical Society, the American Physical Society, and the American Association for the Advancement of Science, he also is a member of German and European physical societies. Dr. Pfau received the 1998 Rudolf-Kaiser Award and in 2014, the Gentner-Kastler Prize of the Société Française de Physique and the German Physical Society. He also received the APS Herbert P. Broida Prize in 2017.
    \end{block}


%%%%%%%%%%%%%%%%%%%%%%%%%%%%%%%%%%%%%%%%%%%%%%%%%%%%%
%            BLOCK: RESEARCH INTERESTS              %
%%%%%%%%%%%%%%%%%%%%%%%%%%%%%%%%%%%%%%%%%%%%%%%%%%%%%
    \begin{block}{Research Interests}
        Prof. Pfau pioneered the cooling and study of dipolar quantum gases starting from the first observation of dipolar effects in a chromium condensate, to the observation of self-bound droplets of a dilute magnetic quantum liquid. He has studied the Rydberg excitation of atoms in a quantum gas, and observed the first ultralong-range Rydberg molecules, as well as investigated the exaggerated properties of this kind of molecule. 
        \begin{center}
	    	\includegraphics[width=0.95\textwidth]{images/ferro}    		
    	\end{center}
    	\textit{Rosensweig instability of a ferrofluid in magnetic field.} \cite{lahayePhysicsDipolarBosonic2009}
    \end{block}
\end{column}
\begin{column}{0.55\textwidth - 1cm}


%%%%%%%%%%%%%%%%%%%%%%%%%%%%%%%%%%%%%%%%%%%%%%%%%%%%%
%                 BLOCK: ABSTRACT                   %
%%%%%%%%%%%%%%%%%%%%%%%%%%%%%%%%%%%%%%%%%%%%%%%%%%%%%
    \begin{block}{Talk Abstract}
         The discovery of high-energy cosmic neutrinos opened a new window of astroparticle physics. Their origin is a new mystery in the field, which is tightly connected to the long-standing puzzle about the origin of cosmic rays. I will discuss theoretical implications of the latest results on high-energy neutrino and cosmic-ray observations, and demonstrate the power of multi-messenger approaches. In particular, I will show that the observed fluxes of neutrinos, gamma rays, and extragalactic cosmic rays can be understood in a unified manner. I will also highlight our recent developments about astrophysical neutrino emission from gamma-ray dark sources and violent transient phenomena such as supermassive black hole flares. I may discuss some possibilities of utilizing high-energy neutrinos as a probe of physics beyond the Standard Model.
    \end{block}


%%%%%%%%%%%%%%%%%%%%%%%%%%%%%%%%%%%%%%%%%%%%%%%%%%%%%
%                BLOCK: BACKGROUND                  %
%%%%%%%%%%%%%%%%%%%%%%%%%%%%%%%%%%%%%%%%%%%%%%%%%%%%%
    \begin{block}{Brief Background}
        Quantum droplets \cite{lahayePhysicsDipolarBosonic2009} 

        \begin{center}
		   	\includegraphics[width=0.95\textwidth]{images/droplet}    		
    	\end{center}
		 Supersolids \cite{bottcherNewStatesMatter2020} 

    \end{block}


%%%%%%%%%%%%%%%%%%%%%%%%%%%%%%%%%%%%%%%%%%%%%%%%%%%%%
%                 BLOCK: REFERENCES                 %
%%%%%%%%%%%%%%%%%%%%%%%%%%%%%%%%%%%%%%%%%%%%%%%%%%%%%
    \begin{block}{References}
        \bibliographystyle{aipnum4-1}
%        \bibliographystyle{iopart-num}
		\bibliography{../references}
    \end{block}

\end{column}
\end{columns}


%%%%%%%%%%%%%%%%%%%%%%%%%%%%%%%%%%%%%%%%%%%%%%%%%%%%%
%                    FOOTER TEXT                    %
%%%%%%%%%%%%%%%%%%%%%%%%%%%%%%%%%%%%%%%%%%%%%%%%%%%%%
\begin{textblock}{0.5}(0.18, 0.94)
    \color{white}
    \sffamily
    \textbf{Eberly College of Science}
    \\
    Department of Physics
\end{textblock}


%%%%%%%%%%%%%%%%%%%%%%%%%%%%%%%%%%%%%%%%%%%%%%%%%%%%%
%                   END TEMPLATE                    %
%%%%%%%%%%%%%%%%%%%%%%%%%%%%%%%%%%%%%%%%%%%%%%%%%%%%%
\end{frame}
\end{document}
